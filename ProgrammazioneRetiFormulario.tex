\documentclass[a4paper]{report}
\usepackage[utf8]{inputenc}
\usepackage[italian]{babel}
\usepackage{hyperref}
\usepackage{float}
\usepackage{xcolor}
\usepackage{amsmath}

\title{\textbf{Programmazione di Reti} \\ \textit{Formulario}}
\author{Luca Casadei - Francesco Pazzaglia - Martin Tomassi}
\date{\today}

\begin{document}
	\maketitle
	\tableofcontents
	\chapter{Ritardi di trasferimento}
	\section{Tempo di trasmissione}
	\begin{itemize}
		\item $T_{\textit{trasmissione}}$ = Tempo di trasmissione $(s)$
		\item $L$ = Lunghezza del pacchetto $(bit)$ \label{st:hlen}
		\item $R$ = Frequenza (capacità) di trasmissione (bit-rate) $(\frac{bit}{s})$ \label{it:capacity}
	\end{itemize}
	Trasferimento di un pacchetto da router \textbf{A} a router \textbf{B}\\\\
	{$T_{\textit{trasmissione}} = \frac{L}{R}$ \label{em:transtime}
	\section{Tempo di propagazione}
	\begin{itemize}
		\item $D$ = Lunghezza del collegamento $(m)$
		\item $v$ = Velocità (ritardo) di propagazione $(\frac{m}{s})$
		\item $\tau$ = Tempo di propagazione $(s)$
	\end{itemize}
	Si può ricavare il tempo di propagazione:\\\\
	$\tau = \frac{D}{v}$\\\\
	Nel caso della suddivisione del canale in $n$ sotto-canali e considerando la lunghezza del canale totale $D$:\\\\ 
	$\tau_{n} = \frac{\tau}{n}$
	\section{Tempo totale}
	Si ricava da:\\\\
	$T_{\textit{tot}} = \tau + T_{\textit{trasmissione}} + T_{\textit{accodamento}} + T_{\textit{elaborazione}}$
	\section{Quantità di bit presenti sul canale}
	Si ricava attraverso:\\\\
	$L = R * T_{\textit{propagazione}}$
	\section{Scenari Cut-Through e Store \& Forward}
	\subsection{Esempio con Store \& Forward}
	Consideriamo $n$ elementi trasmissivi, avremmo $n$ tempi di tramissione: \\\\ 
	$n * T_{trasmissione}$ \\\\
	Consideriamo ora $k$ elementi che introducono latenza per accodamento e ritrasmissione, otteniamo: \\\\
	$k * T_{accodamento}$\\\\
	Con tempo di propagazione fisico $\tau$ \\\\
	$T_{totale} = n * T_{trasmissione} + k * T_{accodamento} + \tau$
	\subsection{Esempio con Cut-Through}
	In questo caso si considera il tempo di accodamento del solo header e non di tutto il pacchetto, sapendo che per trasmettere un pacchetto trascurando eventuali tempi di elaborazione è: $T_{H} + (T - T_{H})$, da questo si ottiene che con header $H$:\\\\
	$T_{totale} = n * T + k * T_{H} + \tau$
	\chapter{Round-Trip-Time (RTT)} \label{cp:rtt}
	\section{Capacità di propagazione}
	\begin{itemize}
		\item $C = R$ come nella sezione precedente \ref{it:capacity}, è la capacità o frequenza di trasmissione.
	\end{itemize}
	\section{RTT}
	\begin{itemize}
		\item $T_{ack} =$ Tempo acknowledge.
	\end{itemize}
	Equivale al $T_{trasmissione}$ più tutti i tempi di elaborazione e di accodamento $T_{accodamento}$, e anche dei tempi di propagazione $\tau$, come nella sezione precedente \ref{em:transtime}, ma considerando tutta la tratta da percorrere sia per l'andata che per il ritorno (in genere si ha $2\tau$ perché va considerata la propagazione di invio e ricezione.).\\\\
	$RTT = T_{trasmissione} + 2\tau + T_{ack}$
	\chapter{Efficienza protocolli "ARQ"}
	\begin{itemize}
		\item $\text{MSS} = $ Maximum segment size (equivalente a L) \ref{st:hlen}.
		\item $\text{SSTRESH} = $ Segment size threshold (soglia).
		\item $\textit{CWND} = $ Congestion window.
		\item $\text{RCWND} = $ Reciever congestion window.
		\item $T_{trasmissione} = \frac{\text{MSS}}{R}$
	\end{itemize}
	Dimensione del file trasferibile nella window (unità di misura MSS) $= \frac{L}{\text{CWND}}$, considerando che la conversione in MSS si ottiene facendo:
	$\frac{\text{RCWND}}{\text{CWND}}$ e la conversione in MSS del threshold:
	$\frac{\text{SSTHRESH}}{\text{CWND}}$, in generale, si ha\\ $\frac{L\text{(byte)}}{\text{CWND (byte)}} = \text{MSS}$.
	\section{Go-back-N}
	\begin{itemize}
		\item $W$ Numero di pacchetti della window (dimensione).
		\item $T$ Tempo per pacchetto.
	\end{itemize}
	Dipende dal rapporto tra RTT \ref{cp:rtt} e la lunghezza della finestra $NT$, consideriamo il tempo di trasmissione $T_{trasmissione}$ e il tempo di propagazione $\tau$:\\
	Dobbiamo avere $WT \geq RTT \implies$ sviluppando i calcoli si ottiene $W>x$ dove $x$ è il numero di pacchetti da mandare prima dell'acknowledge per avere un'efficienza $1$. \\
	La trasmissione è continua quando $WT \geq RTT$ \\ (discontinua quando $WT < RTT$)
	\section{Stop \& Wait}
	\begin{itemize}
		\item Efficienza $\eta = \frac{T_{trasmissione}}{T_{trasmissione} + T_{propagazione}}$
	\end{itemize}
\end{document}